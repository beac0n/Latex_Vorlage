\section{Einleitung}
	Dieses Kapitel enthält Beispiele zum Einfügen von Abbildungen, Tabellen, etc.

	\subsection{Bilder}
		Zum Einfügen eines Bildes, siehe Abbildung \ref{fig:osgi}, wird die \textit{minipage}-Umgebung genutzt, da die Bilder so gut positioniert werden können.

		\vspace{1em}
		\begin{minipage}{\linewidth}
			\centering
			\includegraphics[width=0.7\linewidth]{Bilder/hs_os.png}
			\captionof{figure}[OSGi Architektur]{OSGi Architektur\footnotemark }
			\label{fig:osgi}
		\end{minipage}
		\footnotetext{Quelle: \url{http://www.osgi.org/Technology/WhatIsOSGi}}

	\subsection{Tabellen}
		In diesem Abschnitt wird eine Tabelle (siehe Tabelle \ref{tab:beispiel}) dargestellt.

		\vspace{1em}
		\begin{table}[!h]
			\centering
			\begin{tabular}{|l|l|l|}
				\hline
				\textbf{Name} & \textbf{Name} & \textbf{Name}\\
				\hline
				1 & 2 & 3\\
				\hline
				4 & 5 & 6\\
				\hline
				7 & 8 & 9\\
				\hline
			\end{tabular}
			\caption{Beispieltabelle}
			\label{tab:beispiel}
		\end{table}

		\pagebreak
	\subsection{Auflistung}
		Für Auflistungen wird die \textit{compactitem}-Umgebung genutzt, wodurch der Zeilenabstand zwischen den Punkten verringert wird.
		\begin{compactitem}
			\item Nur
			\item ein
			\item Beispiel.
		\end{compactitem}

	\subsection{Listings}
		Zuletzt ein Beispiel für ein Listing, in dem Quellcode eingebunden werden kann, siehe Listing \ref{lst:arduino}.

		\vspace{1em}
		\begin{lstlisting}[caption=Arduino Beispielprogramm, label=lst:arduino]
			int ledPin = 13;
			void setup() {
				pinMode(ledPin, OUTPUT);
			}
			void loop() {
				digitalWrite(ledPin, HIGH);
				delay(500);
				digitalWrite(ledPin, LOW);
				delay(500);
			}
		\end{lstlisting}

	\subsection{Tipps}
		Die Quellen befinden sich in der Datei \textit{bibo.bib}. Ein Buch- und eine Online-Quelle sind beispielhaft eingefügt. [Vgl. \cite{buch}, \cite{online}]

		Abkürzungen lassen sich natürlich auch nutzen (\ac{OSGi}). Weiter oben im Latex-Code findet sich das Verzeichnis.
		\pagebreak
